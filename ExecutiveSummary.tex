\documentclass[]{article}
\usepackage{lmodern}
\usepackage{amssymb,amsmath}
\usepackage{ifxetex,ifluatex}
\usepackage{fixltx2e} % provides \textsubscript
\ifnum 0\ifxetex 1\fi\ifluatex 1\fi=0 % if pdftex
  \usepackage[T1]{fontenc}
  \usepackage[utf8]{inputenc}
\else % if luatex or xelatex
  \ifxetex
    \usepackage{mathspec}
  \else
    \usepackage{fontspec}
  \fi
  \defaultfontfeatures{Ligatures=TeX,Scale=MatchLowercase}
\fi
% use upquote if available, for straight quotes in verbatim environments
\IfFileExists{upquote.sty}{\usepackage{upquote}}{}
% use microtype if available
\IfFileExists{microtype.sty}{%
\usepackage{microtype}
\UseMicrotypeSet[protrusion]{basicmath} % disable protrusion for tt fonts
}{}
\usepackage[margin=1in]{geometry}
\usepackage{hyperref}
\hypersetup{unicode=true,
            pdftitle={Risk Benefit Analysis for Toast-USB Launch},
            pdfauthor={Allison McCarty, Nicholas Mobley, Nijia Ke},
            pdfborder={0 0 0},
            breaklinks=true}
\urlstyle{same}  % don't use monospace font for urls
\usepackage{graphicx}
% grffile has become a legacy package: https://ctan.org/pkg/grffile
\IfFileExists{grffile.sty}{%
\usepackage{grffile}
}{}
\makeatletter
\def\maxwidth{\ifdim\Gin@nat@width>\linewidth\linewidth\else\Gin@nat@width\fi}
\def\maxheight{\ifdim\Gin@nat@height>\textheight\textheight\else\Gin@nat@height\fi}
\makeatother
% Scale images if necessary, so that they will not overflow the page
% margins by default, and it is still possible to overwrite the defaults
% using explicit options in \includegraphics[width, height, ...]{}
\setkeys{Gin}{width=\maxwidth,height=\maxheight,keepaspectratio}
\IfFileExists{parskip.sty}{%
\usepackage{parskip}
}{% else
\setlength{\parindent}{0pt}
\setlength{\parskip}{6pt plus 2pt minus 1pt}
}
\setlength{\emergencystretch}{3em}  % prevent overfull lines
\providecommand{\tightlist}{%
  \setlength{\itemsep}{0pt}\setlength{\parskip}{0pt}}
\setcounter{secnumdepth}{0}
% Redefines (sub)paragraphs to behave more like sections
\ifx\paragraph\undefined\else
\let\oldparagraph\paragraph
\renewcommand{\paragraph}[1]{\oldparagraph{#1}\mbox{}}
\fi
\ifx\subparagraph\undefined\else
\let\oldsubparagraph\subparagraph
\renewcommand{\subparagraph}[1]{\oldsubparagraph{#1}\mbox{}}
\fi

%%% Use protect on footnotes to avoid problems with footnotes in titles
\let\rmarkdownfootnote\footnote%
\def\footnote{\protect\rmarkdownfootnote}

%%% Change title format to be more compact
\usepackage{titling}

% Create subtitle command for use in maketitle
\providecommand{\subtitle}[1]{
  \posttitle{
    \begin{center}\large#1\end{center}
    }
}

\setlength{\droptitle}{-2em}

  \title{Risk Benefit Analysis for Toast-USB Launch}
    \pretitle{\vspace{\droptitle}\centering\huge}
  \posttitle{\par}
    \author{Allison McCarty, Nicholas Mobley, Nijia Ke}
    \preauthor{\centering\large\emph}
  \postauthor{\par}
      \predate{\centering\large\emph}
  \postdate{\par}
    \date{4/1/2021}


\begin{document}
\maketitle

The sustained popularity of toast as an American breakfast food coupled
with recent Millennial and Gen-Z trends driving the popularity of
gourmet toast has created an exciting opportunity for innovation that
combines technology and toast. Thus, our client, Toast Co., created the
Toast-USB, which would allow amateurs and toast connoisseurs alike to
create culinary-grade toast straight from their computer. The goal of
this investigation was to perform a statistical analysis to help Toast
Co.~to make informed, data-driven decisions regarding the launch of
Toast-USB. After an in-depth investigation, we estimated that the
favorable response rate for the Toast-USB was 12.14\%, which was under
the 24.13\% benchmark that is required to break-even. Based on this
result, we do not recommend that Toast Co.~proceed with the launch of
the Toast-USB for the general population. Instead, Toast Co.~could
consider re-strategizing marketing to target niche demographic groups.

The data for this investigation comes from two studies. The first study
includes data from a market analysis conducted by Toast Co.~from five
metropolitan areas (Toronto, ON; New York City, NY; Philadelphia, PA;
Dallas, TX; and San Francisco, CA). The second study was conducted by
our firm's DOE in the same five metropolitan areas to determine the
likelihood of success of the Toast-USB. Toast Co.'s market analysis
contained 421 observations of 8 variables, while the DOE's campaign
contained 45,211 observations of 17 variables. We performed baseline
data processing on data from the second sample, including using the
median to replace the missing values for the variables representing
balance, campaign, age, and duration. The two samples both contained
data on age, balance, job, education, marital status, mortgage, and
primary phone variables which could be used to synthesize the key
findings from the first and second study and perform a risk-benefit
analysis on continuation of the campaign.

The goal of exploratory data analysis and definitive statistical
analysis was to address three specific aims:

\begin{enumerate}
\def\labelenumi{\arabic{enumi}.}
\item
  Compare the demographic and campaign-specific features from the two
  samples quantitatively and qualitatively.
\item
  Make a binary (Y/N) recommendation regarding launch of the product
  based on a selection of significant demographic predictors and
  logistic modeling.
\item
  Determine an MSRP that will maximize revenue for the product.
\end{enumerate}

In the initial analysis, we compared the demographic features of the
first and second study. We obtained distribution plots and 95\%
confidence intervals for the numeric variables for the numerical
variables common in both studies. Next, we compared the categorical
variables common in both studies, including job, education, marital
status, mortgage and primary phone. The results of the comparison of the
two studies, both for the numerical and categorial predictors, suggest
that samples in the two studies are unlikely from the same population.
We found individuals who are white, male, and have high levels of
education are overrepresented in the first study, while individuals with
a mortgage are underrepresented in the first study. Additionally, we
found that individuals with no personal loan are overrepresented in the
second study. To confirm these conclusions, we performed a likelihood
ratio test to verify that there is significant evidence that response to
buy the product are associated with all the demographic variables in
both studies. The likelihood ratio tests confirmed that the demographic
composition of the first study was significantly different from the
demographic composition of the randomized sample from the second study.

Next, we performed further exploratory analysis on the data from the
second study to gain insights on which demographic variables would have
a meaningful effect on the favorable response rate. We found that age,
personal loan status, default status and education were key demographic
features of interest for the marketing campaign. We generated 15
logistic models representing every combination of the four
aforementioned variables regressed on favorable response. Based on cross
validation using maximum likelihood ratio, the model including age,
personal loan status, and education was selected as the best model for
this data.

Next, we performed a simulation using the model to predict the favorable
response rate to the Toast-USB. Unfortunately, the estimated response
rate for this sample was 12.14\%, which was under the 24.13\% benchmark
that is required to break-even, and further below the 50.36\% favorable
response rate estimated by the first study. Based on this result, we
recommend against the launch of the Toast-USB to the general market.
Instead, Toast Co.~should consider a launch that targets the niche
market of older, highly-educated demographic groups. In addition, Toast
Co.~could also reevaluate their business strategy by making updates to
the product itself, including adding more functional features, updating
the aesthetic design, or improving the software systems to the product
with this demographic group in mind.

Lastly, we performed a quantitative analysis to determine the MSRP for
the Toast-USB to maximize revenue for the company. To do this, we
performed a simulation using the probability that the individual would
purchase the product calculated from the logistic model and the price
that the individual cited that they were willing to pay. By this
process, we determined that the optimal price for this product is 46
USD, which would generate an estimated revenue of 60,000 USD.


\end{document}
